%%%_* Document preamble
\documentclass[14pt,t,usepdftitle=false,xcolornames=x11names,svgnames,dvipsnames,usenames]{beamer}
% \usepackage[T1]{fontenc}
\usepackage{fontspec}% provides font selecting commands
\usepackage{xunicode}% provides unicode character macros
\usepackage{xltxtra} % provides some fixes/extras
\usepackage{listings}
\usepackage{alltt}
\usepackage{varwidth}
\usepackage{setspace}
\usepackage{amsfonts}

\input{texsrc/scala.sty}
%%%_* Simplest theme ever; white background, no widgets
\newcommand{\lmss}{\fontfamily{lmtt}\selectfont\small}
\newcommand{\mlmss}{\fontfamily{lmtt}\selectfont\footnotesize}
\newcommand{\slmss}{\fontfamily{lmtt}\selectfont\scriptsize}
\usetheme{default}
\setbeamertemplate{navigation symbols}{}
\setbeameroption{show notes}
\setbeamertemplate{frametitle}[default][center]
\setbeamersize{text margin left=10mm}
%%%_** Fancy fonts
\setbeamerfont{frametitle}{}
\setmainfont[Mapping=tex-text]{Latin Modern Sans}
\setsansfont[Mapping=tex-text,Numbers={OldStyle}]{Latin Modern Sans}
\setmonofont[Mapping=tex-text]{Latin Modern Mono}
\newcommand{\wackyFont}[1]{
  {\LARGE\fontspec[Mapping=tex-text]{Trebuchet MS} #1}}
\newcommand{\wackyFontN}[1]{
  {\fontspec[Mapping=tex-text]{Trebuchet MS} #1}}
\newcommand{\includelisting}[1]{
  {\lmss\input{#1}}}
\newcommand{\sincludelisting}[1]{
  {\slmss\input{#1}}}
\newcommand{\sexample}[1]{
  \colorbox{LightGrey}{\begin{varwidth}{\textwidth}{\slmss\begin{spacing}{0.5}\input{#1}\end{spacing}}\end{varwidth}}}
\newcommand{\cexample}[1]{
  {\slmss\begin{spacing}{0.5}\input{texsrc/#1.scala.tex}\end{spacing}}}
\newcommand{\tinyurl}[1]{
  {\tiny{\textcolor{keyword}{\url{#1}}}}}
\newcommand{\featureexample}[2]{
  {\small{Example: \emph{#1}}\vskip-4mm\hrulefill\vskip+3mm\cexample{#2}}}
\newcommand{\featurecategory}[2]{
  {\begin{center}\wackyFontN{#1}\end{center}\vskip-4mm#2}}
\newcommand{\feature}[1]{
  {\begin{center}#1\end{center}}}
\newcommand{\featureframe}[3]{
  {\begin{frame}{#1}
     \featurecategory{#2}{#3}
   \end{frame}}}


\newcommand{\mincludelisting}[1]{
  {\mlmss\input{#1}}}
\newcommand{\hilight}[1]{
  {\textcolor{hilite}{\emph{#1}}}}
\newcommand{\subtitleFont}[1]{{\footnotesize #1}}
\newcommand{\slideheading}[1]{
  \begin{center}
    \usebeamerfont{frametitle}
    \usebeamercolor[fg]{frametitle}#1
  \end{center}\vskip-5mm}
\usefonttheme{professionalfonts}
%%%_** Color definitions
\colorlet{comment}{Olive}
\colorlet{string}{SaddleBrown}
\colorlet{keyword}{Navy}
\colorlet{type}{Green}
\colorlet{emph}{Maroon}
\colorlet{input}{Indigo}
\colorlet{error}{DarkRed}
\colorlet{intermediate}{LightSlateGrey}
\colorlet{result}{LightSlateGrey}
\colorlet{hilite}{Red}
\colorlet{background}{LightGoldenrodYellow}
\colorlet{hole}{LimeGreen}

%%%_* Document
\begin{document}

\title{\wackyFont{Scala Clinic}}
\subtitle{\textbf{Intermediate Scala}}
\author{Jim~Powers\\\subtitleFont{Patch.com}}
\date{\subtitleFont{23 February 2012}}

\maketitle

\section{Intermediate Scala}

\begin{frame}{Intermediate Scala}
  \begin{itemize}[<+->]
    \item Implicits
    \begin{itemize}
      \item Implicit Conversions
      \item Implicit Arguments
    \end{itemize}
    \item Type Bounds
    \item Views (collections)
    \item View Bounds
    \item Context Bounds
    \item \hilight{Type Classes}
    \item Manifests and Type Erasure
    \item Higher-Kinded Types
  \end{itemize}
\end{frame}

\begin{frame}{Implicits}
  \hilight{Implicits} are a way to tell the compiler of a way to prove (\emph{provide evidence}) that there is a way to get from \emph{here} (desired functionality) to \emph{there} (actual functionality) with the compiler actually \emph{connecting the dots} for you.
\end{frame}

\begin{frame}{Detour: \emph{Companion Objects}}
  \hilight{Companion Objects} are objects that share the same name as a \emph{class} in the same compilation unit.  \emph{Companion Objects} can be used to house implicit definitions to be used whenever the associated class is in scope.
\end{frame}

\featureframe{Companion Objects}
             {Companion objects can hold stuff for instances}
             {\featureexample{}{companion}}

\begin{frame}{Implicit Scope}
  \hilight{Implicit Scope} is the \emph{search space} used by the compiler to resolve requests for \emph{implicit evidence}.
\end{frame}

\begin{frame}{Implicit Scope}
  \begin{itemize}[<+->]
    \item First look in current scope
    \begin{itemize}
      \item Implicits defined in current scope (1)
      \item Explicit imports (2)
      \item wildcard imports (3)
      \item Same scope in other files (4)
    \end{itemize}
    \item Parts of the type of implicit value being looked up and their companion objects
    \begin{itemize}
      \item Companion objects of the type
      \item Companion objects of type arguments of types
      \item Outer objects for nested types
      \item Other dimensions
    \end{itemize}
  \end{itemize}
  \begin{center}
    \tiny{\textcolor{keyword}{Taken from Josh Suereth's NE Scala Symposium presentation:\\}}\tinyurl{https://docs.google.com/present/view?id=dfqn4jb_106hq4mvbd8}
  \end{center}
\end{frame}

\begin{frame}{Implicit Conversions}
  \hilight{Implicit Conversions} provide a way to make \emph{ad-hoc} conversions between values of some type.  There are two primary use-cases for \emph{implicit conversions}:
\end{frame}

\begin{frame}{Implicit Conversions}
  \hilight{Implicit Conversions} provide a way to make \emph{ad-hoc} conversions between values of some type.  There are two primary use-cases for \emph{implicit conversions}:
  \begin{itemize}[<+->]
    \item To ``pimp'' or extend existing types with new functionality
    \begin{itemize}
      \item One of the ways to ``lift'' values into a \emph{DSL}
    \end{itemize}
    \item To remove conversion boilerplate when a value one type can unambiguously be used \emph{as} value of another type.
  \end{itemize}
\end{frame}

\featureframe{Implicits}
             {``Pimping''}
             {\featureexample{}{pimped}}

\featureframe{Implicits}
             {Converting}
             {\featureexample{}{conversion}}

\begin{frame}{Implicit Arguments}
  \hilight{Implicit Arguments} enable the compiler to supply arguments based on type alone (although you can supply your own arguments as well).  The feature effectively allows the compiler to infer \emph{proofs}.  The implicit argument mechanism enables the \hilight{typeclass pattern}.
\end{frame}

\featureframe{Implicits}
             {Implicit Arguments}
             {\featureexample{}{implicit_arguments}}

\begin{frame}{Type Bounds}
  \hilight{Type Bounds} provide a way to constrain the types supplied as type parameters.
\end{frame}

\begin{frame}{Detour: Variance}
  \hilight{Variance} describes the acceptable \emph{variations} of a type parameter when relating two values.  The options are:
\end{frame}

\begin{frame}{Detour: Variance}
  \hilight{Variance} describes the acceptable \emph{variations} of a type parameter when relating two values.  The options are:
  \begin{itemize}[<+->]
    \item \hilight{Invariant} - type parameters must match \emph{exactly}
    \item \hilight{Covariant} - acceptable values can be of the same type or a \emph{subtype}
    \item \hilight{Contravariant} - acceptable values can be of the same type or a \emph{supertype}
  \end{itemize}
\end{frame}

\featureframe{Detour: Varaince}
             {Invariance}
             {\featureexample{}{invaraince_example}}

\featureframe{Detour: Variance}
             {Covaraince}
             {\featureexample{}{covariance_example}}

\featureframe{Detour: Variance}
             {Contravaraince}
             {\featureexample{}{contravaraince_example}}

\begin{frame}{Type Bounds}
  \hilight{Type Bounds} are just ways to further constrain \emph{variance}!
\end{frame}

\featureframe{Type Bounds}
             {Upper Bound}
             {\featureexample{}{upper_bound}}

\featureframe{Type Bounds}
             {Lower Bound}
             {\featureexample{}{lower_bound}}

\featureframe{Type Bounds}
             {Upper and Lower Bound}
             {\featureexample{}{upper_and_lower_bound}}

\begin{frame}{Views (collections)}
  \hilight{Views on collections} is a way to \emph{fuse} transformer operations such as {\tt map, flatMap, filter, etc.} into a single operation and apply that operation once to a collection avoiding the creation of intermediate collections.  Intelligent use of collection views can significantly improve performance while not losing the expressiveness of Scala's collections library.
\end{frame}

\featureframe{Views (collections)}
             {Bad pattern}
             {\featureexample{}{no_view}}

\featureframe{Views (collections)}
             {Better pattern}
             {\featureexample{}{view}}

\begin{frame}{View Bounds (or just plain Views)}
  \hilight{View Bounds} are a way to express to the compiler a requirement that a type $T$ be treated as a type $S$ in some scope.  This requirement can be fulfilled by the compiler if an \emph{implicit conversion} can be found.  The type of the implicit conversion has to be $T \Rightarrow S$.
\end{frame}

\featureframe{View Bounds}
             {Sugared}
             {\featureexample{}{sugared}}

\featureframe{View Bounds}
             {De-sugared}
             {\featureexample{}{de-sugared}}

\begin{frame}{Context Bounds}
  \hilight{Context Bounds} is a way to tell the compiler to look for a \emph{correspondence} to a given type in implicit scope and make that correspondence implicit in a newly introduced scope.
\end{frame}

\begin{frame}{Context Bounds: More technical explanation}
  \hilight{Context Bounds} defines a constraint on a type $T$ that says that a corresponding value of type {\tt C[$T$]} must be visible in implicit scope to successfully type-check.  If such a correspondence exists then introduce the corresponding value of type {\tt C[$T$]} into a newly defined scope.
\end{frame}

\featureframe{Context Bounds}
             {Sugared}
             {\featureexample{}{cb_sugared}}

\featureframe{Context Bounds}
             {De-sugared}
             {\featureexample{}{cb_de-sugared}}

\featureframe{Context Bounds}
             {Defining ``Zeros''}
             {\featureexample{}{cb_zeros}}

\begin{frame}{Type Classes}
  The term \hilight{Type Classes} comes from research into \emph{typed lambda calculus} and has been made popular by the programming language \emph{Haskell}.  Even though the word ``classes'' appears in the term, it has \emph{nothing, whatsoever, to do with OO or Scala classes}. \hilight{Type Classes} are a way to address the need for \emph{ad-hoc polymorphism} in a language.
\end{frame}

\begin{frame}{Type Classes}
  \hilight{Type Classes} are a considerably more powerful way to construct software systems than by using structural inheritance.
\end{frame}

\featureframe{Type Classes}
             {Appending stuff}
             {\featureexample{Lots of duplication}{bad_append}}

\begin{frame}{Type Classes}
  But I just want a generic append function!  After some research you discover the notion of a \hilight{semigroup}.
\end{frame}

\featureframe{Type Classes}
             {Semigroup}
             {\featureexample{}{semigroup}}

\begin{frame}{Type Classes}
  \hilight{Semigroups} are close to what we want, but we need to get $T$ into $M[T]$ in some generic fashion.  More research turns up the \hilight{monoid}.
\end{frame}

\featureframe{Type Classes}
             {Monoid}
             {\featureexample{}{monoid}}

\featureframe{Type Classes}
             {Monoid Typeclass}
             {\featureexample{}{monoid_typeclass}}

\featureframe{Type Classes}
             {Generalized Monoids}
             {\featureexample{}{generalized_monoids}}

\featureframe{Type Classes}
             {Generalized Monoids}
             {\featureexample{Choose your monoid}{choose_monoids}}

\begin{frame}{Manifests and Type Erasure}
  Sadly, not all type information available at compile time is not available at run time.  In particular, type parameter information is lost at run time.
\end{frame}

\featureframe{Manifests and Type Erasure}
             {Type information lost}
             {\featureexample{}{erasure}}

\featureframe{Manifests and Type Erasure}
             {Recovering type information}
             {\featureexample{}{manifests}}

\begin{frame}{Higher-Kinded Types}
  \hilight{Types} classify values, \hilight{kinds} classify types.
\end{frame}

\begin{frame}{Higher-Kinded Types}
  \begin{center}
    Kind: $*$
  \end{center}
  \vskip-4mm\hrulefill\vskip+3mm
  \begin{itemize}[<+->]
    \item Int, Byte, String, ...
    \item Classes with no type parameters
    \item List[Int], Stream[String], ...
    \item Types with parameters where all parameters are filled in
  \end{itemize}
\end{frame}

\begin{frame}{Higher-Kinded Types}
  \begin{center}
    Kind: $* \to *$
  \end{center}
  \vskip-4mm\hrulefill\vskip+3mm
  \begin{itemize}[<+->]
    \item List,Stream,Vector,set
    \item Any type with a single type-parameter argument
  \end{itemize}
\end{frame}

\begin{frame}{Higher-Kinded Types}
  \begin{center}
    Kind: $* \to * \to *$
  \end{center}
  \vskip-4mm\hrulefill\vskip+3mm
  \begin{itemize}[<+->]
    \item Function1, Map, PartialFunction
    \item Any type with a two type-parameters
  \end{itemize}
\end{frame}

\end{document}