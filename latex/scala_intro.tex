%%%_* Document preamble
\documentclass[14pt,t,usepdftitle=false,xcolornames=x11names,svgnames,dvipsnames,usenames]{beamer}
% \usepackage[T1]{fontenc}
\usepackage{fontspec}% provides font selecting commands
\usepackage{xunicode}% provides unicode character macros
\usepackage{xltxtra} % provides some fixes/extras
\usepackage{listings}
\usepackage{alltt}
\usepackage{varwidth}
\usepackage{setspace}

\input{texsrc/scala.sty}
%%%_* Simplest theme ever; white background, no widgets
\newcommand{\lmss}{\fontfamily{lmtt}\selectfont\small}
\newcommand{\mlmss}{\fontfamily{lmtt}\selectfont\footnotesize}
\newcommand{\slmss}{\fontfamily{lmtt}\selectfont\scriptsize}
\usetheme{default}
\setbeamertemplate{navigation symbols}{}
\setbeameroption{show notes}
\setbeamertemplate{frametitle}[default][center]
\setbeamersize{text margin left=10mm}
%%%_** Fancy fonts
\setbeamerfont{frametitle}{}
\setmainfont[Mapping=tex-text]{Latin Modern Sans}
\setsansfont[Mapping=tex-text,Numbers={OldStyle}]{Latin Modern Sans}
\setmonofont[Mapping=tex-text]{Latin Modern Mono}
\newcommand{\wackyFont}[1]{
  {\LARGE\fontspec[Mapping=tex-text]{Trebuchet MS} #1}}
\newcommand{\wackyFontN}[1]{
  {\fontspec[Mapping=tex-text]{Trebuchet MS} #1}}
\newcommand{\includelisting}[1]{
  {\lmss\input{#1}}}
\newcommand{\sincludelisting}[1]{
  {\slmss\input{#1}}}
\newcommand{\sexample}[1]{
  \colorbox{LightGrey}{\begin{varwidth}{\textwidth}{\slmss\begin{spacing}{0.5}\input{#1}\end{spacing}}\end{varwidth}}}
\newcommand{\cexample}[1]{
  {\slmss\begin{spacing}{0.5}\input{texsrc/#1.scala.tex}\end{spacing}}}
\newcommand{\tinyurl}[1]{
  {\tiny{\textcolor{keyword}{\url{#1}}}}}
\newcommand{\featureexample}[2]{
  {\small{Example: \emph{#1}}\vskip-4mm\hrulefill\vskip+3mm\cexample{#2}}}
\newcommand{\featurecategory}[2]{
  {\begin{center}\wackyFontN{#1}\end{center}\vskip-4mm#2}}
\newcommand{\feature}[1]{
  {\begin{center}#1\end{center}}}
\newcommand{\featureframe}[3]{
  {\begin{frame}{Language Features: #1}
     \featurecategory{#2}{#3}
   \end{frame}}}


\newcommand{\mincludelisting}[1]{
  {\mlmss\input{#1}}}
\newcommand{\subtitleFont}[1]{{\footnotesize #1}}
\newcommand{\slideheading}[1]{
  \begin{center}
    \usebeamerfont{frametitle}
    \usebeamercolor[fg]{frametitle}#1
  \end{center}\vskip-5mm}
\usefonttheme{professionalfonts}
%%%_** Color definitions
\colorlet{comment}{Olive}
\colorlet{string}{SaddleBrown}
\colorlet{keyword}{Navy}
\colorlet{type}{Green}
\colorlet{emph}{Maroon}
\colorlet{input}{Indigo}
\colorlet{error}{DarkRed}
\colorlet{intermediate}{LightSlateGrey}
\colorlet{result}{LightSlateGrey}
\colorlet{hilite}{Red}
\colorlet{background}{LightGoldenrodYellow}
\colorlet{hole}{LimeGreen}

%%%_* Document
\begin{document}

\title{\wackyFont{Scala Clinic}}
\subtitle{\textbf{Scala basics...}}
\author{Jim~Powers\\\subtitleFont{Patch.com}}
\date{\subtitleFont{8 February 2012}}

\maketitle

\section{What is Scala?}

\begin{frame}{What is Scala?}
  \uncover<2->{Scala is a multi-paradigm programming language designed to
    integrate features of object-oriented programming and functional
    programming. The name Scala is a portmanteau of \textcolor{hilite}{\emph{scalable}} and
    \textcolor{hilite}{\emph{language}}, signifying that it is designed to grow with the
    demands of its users. James Strachan, the creator of Groovy,
    described Scala as a possible successor to Java.\\
    \vspace{7mm}
    \begin{center}
      \tiny{\textcolor{keyword}{\url{http://en.wikipedia.org/wiki/Scala_(programming_language)}}}}
    \end{center}
\end{frame}

\begin{frame}{Why Scala?}
  \begin{itemize}[<+->]
    \item It's not Java?
    \item Inter-operates with Java eco-system well
    \item Many features to aid building large systems
    \item Good performance
    \item Probably the best language on the JVM
    \item Getting better all the time
  \end{itemize}
\end{frame}

\begin{frame}{Why JVM?}
  \begin{itemize}[<+->]
    \item Tons of stuff written for it (some of it good)
    \item Fewer surprises running code in different environments
    \item Good performance (thousands of man-years)
  \end{itemize}
\end{frame}

\begin{frame}{My Opinions on Scala}
  \begin{itemize}[<+->]
    \item It's mostly nice, kinda. - I'm not a rabid fanboi
    \item Better than Java
    \item Better than most alternatives on the JVM
    \item Good run-time performance
    \item Smart people working on the language and tools
    \item Compiler is slow
    \item Will be happier when macros land
  \end{itemize}
\end{frame}

\begin{frame}{Getting Started}
  \begin{itemize}[<+->]
    \item Get Java JDK 1.6 (1.7 if you want to experiment)
    \item Download it (2.9.1 current) \tinyurl{http://scala-lang.org}
    \item Install SBT \tinyurl{https://github.com/harrah/xsbt}
    \item TextMate/Sublime Text/Vim/Emacs/JEdit
    \item Intelli/J IDEA
    \item Eclipse
    \item Plays well with most JVM tools
  \end{itemize}
\end{frame}

\begin{frame}{On-Line Learning Resources}
  \begin{itemize}[<+->]
    \item Scala Documentation Project \tinyurl{http://docs.scala-lang.org/}
    \item Official Scala Web Site \tinyurl{http://scala-lang.org}
    \item Scala-user Google Group \tinyurl{https://groups.google.com/forum/?fromgroups\#!forum/scala-user}
    \item Stackoverflow \tinyurl{http://stackoverflow.com/tags/scala/info}
    \item Scala Koans \tinyurl{https://bitbucket.org/dickwall/scala-koans}
    \item Debasish Ghosh's Blog \tinyurl{http://debasishg.blogspot.com/}
    \item Tony Morris' Blog \tinyurl{http://blog.tmorris.net/}
    \item Jim McBeath's Blog \tinyurl{http://jim-mcbeath.blogspot.com/}
  \end{itemize}
\end{frame}

\begin{frame}{Books}
  \begin{itemize}[<+->]
    \item A pretty comprehensive list can be found here \tinyurl{http://www.scala-lang.org/node/959}
    \item Martin Odersky's Book \tinyurl{http://www.artima.com/shop/programming_in_scala_2ed}
    \item Josh Seureth's Book \tinyurl{http://www.manning.com/suereth/}
  \end{itemize}
\end{frame}

\begin{frame}{Community}
  \begin{itemize}[<+->]
    \item Google Groups (there are a bunch)
    \item Scala Documentation Project \tinyurl{http://docs.scala-lang.org/}
    \item Scala Improvemet Process \tinyurl{http://docs.scala-lang.org/sips/index.html}
    \item IRC: \#scala on freenode
    \item Meetups/Hackathons
  \end{itemize}
\end{frame}

\begin{frame}{Frame of Mind}
  \begin{quote}
    It is a logical impossibility to make a language more powerful
    by omitting features, no matter how bad they may be.
  \end{quote}
  \begin{flushright}
    \tiny{\textcolor{keyword}{-- John Huges, \emph{Why Functional Programming Matters}, 1990}}
  \end{flushright}
\end{frame}

\begin{frame}{Frame of Mind}
  \begin{quote}
    Scala seems designed on the principle that if we can't have nice
    things, we can at least have lots and lots of meh ones.
  \end{quote}
  \begin{flushright}
    \tiny{\textcolor{keyword}{-- Bryan O'Sullivan, \url{https://twitter.com/\#!/bos31337/status/155102828774428672}}}
  \end{flushright}
\end{frame}

\section{Language Features}

\begin{frame}{Language Features}
  \begin{itemize}
    \item Strongly Typed
      \begin{itemize}[<+->]
        \item Statically typed
        \item Local type inference\\\sexample{texsrc/local_type_inference.scala.tex}
        \item Parameterized types (generics)\\\sexample{texsrc/parameterized_types.scala.tex}
        \item Type aliases\\\sexample{texsrc/type_aliases.scala.tex}
        \item \textcolor{intermediate}{``Higher-kinded'' Types}
        \item \textcolor{intermediate}{Specialized Generics}
      \end{itemize}
  \end{itemize}
\end{frame}

\begin{frame}{Language Features}
  \begin{itemize}
    \item Common Types
      \begin{itemize}[<+->]
        \item Primatives: {\lmss Byte, Int, Long, Short, Double, Float, Char, Boolean}
        \item Arrays: {\lmss Array[\_](size)}
        \item Strings
        \item Functions\\\sexample{texsrc/functions.scala.tex}
        \item List\\\sexample{texsrc/List.scala.tex}
      \end{itemize}
  \end{itemize}
\end{frame}

\begin{frame}{Language Features}
  \begin{itemize}
    \item Common Types
      \begin{itemize}[<+->]
        \item Vector: {\lmss Vector[\_](v$_0$,v$_1$,v$_3$,\ldots)}
        \item Map: {\lmss Map[\_,\_](k$_0$ -> v$_0$,k$_1$ -> v$_1$,k$_3$ -> v$_3$,\ldots)}
        \item Set: {\lmss Set[\_](k$_0$,k$_1$,k$_3$,\ldots)}
        \item Stream\\\sexample{texsrc/Stream.scala.tex}
        \item Unit:{\lmss ()}
        \item Tuples \emph{a maximum size of 22}\\\sexample{texsrc/Tuples.scala.tex}
      \end{itemize}
  \end{itemize}
\end{frame}

\begin{frame}{Language Features}
  \begin{itemize}
    \item Common Types
      \begin{itemize}[<+->]
        \item Option\\\sexample{texsrc/Option.scala.tex}
      \end{itemize}
  \end{itemize}
\end{frame}

\begin{frame}{Language Features}
  \begin{itemize}
    \item Common Types
      \begin{itemize}[<+->]
        \item Either\\\sexample{texsrc/Either.scala.tex}
      \end{itemize}
  \end{itemize}
\end{frame}

\featureframe{Object-Oriented}
             {Classes and Traits}
             {\featureexample{Simple}{simple_classes}}

\featureframe{Object-Oriented}
             {Classes and Traits}
             {\featureexample{Single inheritance}{single_inheritance}}

\featureframe{Object-Oriented}
             {Classes and Traits}
             {\feature{Traits do not have constructors}}

\featureframe{Object-Oriented}
             {Classes and Traits}
             {\featureexample{Classes have canonical constructor}{constructors}}

\featureframe{Object-Oriented}
             {Classes and Traits}
             {\featureexample{Classes have canonical constructor}{constructors1}}

\featureframe{Object-Oriented}
             {Classes and Traits}
             {\featureexample{Classes can have alternate constructors}{alternate_constructors}}

\featureframe{Object-Oriented}
             {Classes and Traits}
             {\featureexample{Classes and Traits can be nested}{nested_classes}}

\featureframe{Object-Oriented}
             {Classes and Traits}
             {\featureexample{Classes and Traits can be nested}{nested_traits}}

\featureframe{Object-Oriented}
             {Classes and Traits}
             {\featureexample{Traits can carry implementations}{sample_traits}}

\featureframe{Object-Oriented}
             {Classes and Traits}
             {\featureexample{Mixins}{mixins1}\small{The principle hack used to implement the \emph{Cake Pattern}}}

\featureframe{Object-Oriented}
             {Classes and Traits}
             {\featureexample{Ad-hoc Enrichment}{ad_hoc}}

\featureframe{Object-Oriented}
             {Structural Types}
             {\featureexample{Structural Types}{structural}}

\featureframe{Object-Oriented}
             {Anonymous Classes}
             {\featureexample{Anonymous Classes}{anonymous_classes}}

\featureframe{Object-Oriented}
             {Self-naming}
             {\featureexample{Self-naming}{self_naming}}

\featureframe{Object-Oriented}
             {Traits can Self-type}
             {\featureexample{Self-typing}{self_typing}
                    \small{Why would you do this? -- self-types constrain
                           inheritance without exposing an is-a relationship.}}

\featureframe{Object-Oriented}
             {First-class Modules}
             {\featureexample{Classes as Modules}{classes_as_modules}}

\featureframe{Object-Oriented}
             {First-class Modules}
             {\featureexample{Objects as Modules}{objects_as_modules}}

\featureframe{Object-Oriented}
             {First-class Modules}
             {\featureexample{Package Objects}{package_objects}}

\featureframe{Object-Oriented}
             {First-class Modules}
             {\featureexample{Visibility Control}{visibility_control}}

\featureframe{Object-Oriented}
             {First-class Modules}
             {\featureexample{Imports}{imports}}

\featureframe{Object-Oriented}
             {Simulated ``Algebraic Data Types''}
             {\featureexample{}{algebraic_data_types1}}

\featureframe{Object-Oriented}
             {Simulated ``Algebraic Data Types''}
             {\featureexample{Nifty Trick}{algebraic_data_types2}}

\featureframe{Object-Oriented}
             {Case Classes}
             {\featureexample{Copy Synthetic}{copy_example}}

\featureframe{Functional}
             {First-class functions!}
             {\featureexample{}{first_class_functions}}

\featureframe{Functional}
             {Function ``Objects''}
             {\featureexample{}{function_objects}}

\featureframe{Functional}
             {Partial Functions}
             {\featureexample{}{partial_functions1}}

\featureframe{Functional}
             {Partial Functions}
             {\featureexample{Partial function chaining}{partial_functions2}}

\featureframe{Functional}
             {Partial Functions}
             {\featureexample{Guarded}{partial_functions3}}

\featureframe{Functional}
             {Parameter Groups}
             {\featureexample{}{parameter_groups}}

\featureframe{Functional}
             {By Reference, Value and Name arguments}
             {\featureexample{}{arguments}}

\featureframe{Functional}
             {Variable-sized Argument Lists}
             {\featureexample{}{variable_arguments}}

\featureframe{Functional}
             {Methods and Functions}
             {\featureexample{Methods}{methods_vs_functions}}

\featureframe{Functional}
             {Methods and Functions}
             {\featureexample{Functions}{methods_vs_functions1}}

\featureframe{Other}
             {Var and Val}
             {\featureexample{}{var_vs_val}}

\featureframe{Other}
             {Lazy Values}
             {\featureexample{}{lazy_example}}

\featureframe{Other}
             {If expressions}
             {\featureexample{}{if_example}}

\featureframe{Other}
             {While loops}
             {\featureexample{}{while_example}}

\featureframe{Other}
             {For-comprehensions}
             {\featureexample{Monadic For}{monadic_for}}

\featureframe{Other}
             {For-comprehensions}
             {\featureexample{Imperative For}{imperative_for}}

\featureframe{Other}
             {Pattern-matching}
             {\featureexample{Value Matching}{matching1}}

\featureframe{Other}
             {Pattern-matching}
             {\featureexample{``Otherwise'' Match}{matching2}}

\featureframe{Other}
             {Pattern-matching}
             {\featureexample{Alternates Match}{matching3}}

\featureframe{Other}
             {Pattern-matching}
             {\featureexample{Extractor Match}{matching4}}

\featureframe{Other}
             {Pattern-matching}
             {\featureexample{Mixed Extractor/Value Match}{matching5}}

\featureframe{Other}
             {Pattern-matching}
             {\featureexample{Match with Guard}{matching6}}

\featureframe{Other}
             {Pattern-matching}
             {\featureexample{Trick for matching multiple criteria}{matching7}}

\featureframe{Other}
             {Pattern-matching}
             {\featureexample{``@'' Matching}{matching8}}

\end{document}
